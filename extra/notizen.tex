Im Lauf des Tages ändert das Verkehrsaufkommen seine Richtung. Deshalb koordiniert Berlin die Ampeln auf vielen Ausfallstraßen mit der Tageszeit. Wer morgens Richtung Innenstadt unterwegs ist, hat häufiger Grün, als der, der stadtauswärts fährt. Am frühen Abend ist es dafür umgekehrt. Weil morgens mehr Menschen in die Stadt und abends mehr aus der Stadt fahren.

http://www.stadtentwicklung.berlin.de/verkehr/lenkung/ampeln/sicherheit/de/gruene_welle.shtml\\\\

Auf vielen Straßen und Kreuzungen sollen Busse und Straßenbahnen Vorfahrt haben.


An vielen Berliner Kreuzungen können Fahrzeuge des ÖPNV über ein Funksignal die Signalfarbe Grün anfordern. An besonders prekären Verkehrsknotenpunkten hat der ÖPNV-Verkehr außerdem eigene, weiß leuchtende Bus- und Bahnampeln.
Etwa 600 Meter vor der Ampel sendet der Bus oder die Straßenbahn ein Voranmeldesignal über Funk. Daraufhin prüft das Steuergerät der Ampel, welches Signal die Ampel derzeit zeigt, und bereitet entsprechende Reaktionen vor.\\
Etwa 200 Meter vor der Haltlinie folgt die Hauptanmeldung. Jetzt entscheidet das Steuergerät, ob die Ampel umspringt oder das bestehende Signal verlängert wird, um der Bahn oder dem Bus Vorrang zu gewähren.\\
Haben Bahn oder Bus die Ampel passiert, sorgt ein Abmeldesignal dafür, dass diese wieder ihren Normalbetrieb aufnimmt.
Bis Mitte 2009 wurden bereits rund 800 Ampeln mit Beeinflussungsmöglichkeiten durch Busse und Straßenbahnen ausgerüstet. Das entspricht gut 60\% aller Ampeln mit ÖPNV-Verkehr. Die Straßenbahn hat schon heute an nahezu allen Ampeln Beeinflussungsmöglichkeiten. Derzeit befindet sich das Beschleunigungsprogramm für die Busse in der Umsetzung. Das Ziel ist es, in den nächsten Jahren linienweise weitere Busstrecken zu beschleunigen und entsprechend weitere rund 500 Ampeln auszurüsten.\\
http://www.stadtentwicklung.berlin.de/verkehr/lenkung/ampeln/bus_bahn/index.shtml
