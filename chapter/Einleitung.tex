\chapter{Einführung}
\section{Motivation}
Im Berliner Verkehrswesen ist ein deutlicher Trend zu bemerken. Das Fahrrad wird zum ökologischen und gesundheitlichen, aktiven Lebensstil und wird dem hohen Verkehrsaufkommen der Automobile, insbesondere in der Stadtregion, entgegenwirken. “Fahrradfahren boomt in Berlin stärker als bislang bekannt”\cite{Mopo}\\\\
Neue Fahrradwege und Vergrößerung des Fahrradstraßennetzes sind regionale Baumaßnahmen, die dabei aktuell diesen Fahrradtrend unterstützen. Grund der neuen Fahrradeuphorie ist nicht zuletzt die erfolgreiche Etablierung der E-Bikes\footnote{ Elektrofahrrad. Ein Fahrrad mit elektrischem Hilfsmotor}. 
E-Bikes erfreuen sich großer Beliebtheit und ermöglichen auch längere Touren ohne große Anstrengung.\\ 
Die Digitalisierung der Autoinnenräume mit Navigation und Bordelektronik sowie die Verbindungen zu Smartphones stellen aktuell keine Besonderheit mehr dar. Wird das Fahrrad nun als „vollwertiges“ Mitglied im Straßenverkehr angesehen, kann zusätzliche Elektronik wie Navigation und Blickmechanismen die FahrradfahrerInnen unterstützen.\\\\ 
Der Fahrtfluss des Radfahrers soll nicht unnötig unterbrochen werden. Dafür werden die potentiellen Wartezeiten an der nächsten Ampel vorzeitig errechnet und dem Fahrer mitgeteilt. Resultierend kann der Nutzer die Geschwindigkeit anpassen und die verbleibende Wegstrecke zur Ampel nutzen, um bei Grün ohne anzuhalten die Kreuzung zu überqueren. Für die Datenerhebung werden zugleich die mobilen Systeme der Radfahrer genutzt.
\section{Zielstellung}
Um die Ampeldaten zu erfassen, gibt es verschiedene Möglichkeiten. Eine 100prozentige Deckung erreicht man nicht einmal durch manuelle Ablesung jeder Ampel, da circa 20 Prozent der \gls{LSA} in Berlin manuell gesteuert werden. Wenn man das mit Ampeln auf gegebener Teststrecke umsetzt, kann zunächst der Prototyp des Ampelhinweissystem entwickelt werden. \\
Die Auswertung erfolgt entweder durch eine Smartphone-App oder durch ein \gls{LED}-Licht-System per Blinkfrequenz; beides am Lenker angebracht. Bei Nutzung des Telefons, nimmt man den integrierten \gls{GPS}-Sender, bei der zweiten Variante muss man das System mit einem ausstatten.
\\\\
Das Ziel der Arbeit ist ein Konzept und dessen prototypische Anwendung eines Ampelhinweissystem, welches einem auf Basis der zu erstellenden  Ampeldatenbank Informationen über die Ampelschaltung zukommen lässt und ihn so interaktiv durch das Verkehrsnetz führt.
\section{Aufbau der Arbeit}
2. state of the art:\\
Was gibt es schon. Projekte und Studien und fertige Apps,\\ 
3. Grundbegriffe (technische Grundlagen), Theoriewissen (Berechnungen, Formeln etc) Definitionen, Überblick über mögliche Einsatzgebiete \\
4. Analysekapitel:?\\
Anforderungsanalyse für Fahrradapp. Personas werden eingeführt?, =fiktive Benutzer, dann Zusammenfassung der herausgearbeiteten Anforderungen.\\
Funktionalität, graphische Oberfläche
5. Kapitel = Kern der Arbeit -- Konzipierung:\\
mobile Anwendung. app. anrduino. die nutzer auf ampeln hinweist und die dauer der phase. 
es wird auf.. einngegangen. Der Einsatz von mobile sensig wird dargelegt, alles vorgestellt. Zusammenfassend wird das Konzept am Ende von allen Personas nocheinmal kritisch betrachtet und evaluiert.\\
6. Kapitel: Umsetzung in exemplarischen Prototyp der ... Nach klärung der theoretischen berechnungsgrundlagen... dann wird der Prototyp in Design, Funktionalität und Architektur erläutert und schließlich in mehreren Testreihen uf die Probe gestellt. Die Ergebnisse folgen in den letzten Abschnitten des Kapitels.\\
7. Kapitel: \\
Abschluss dieser Arbeit = Evaluation der These der Arbeit, Zusammenfassung, Auslick auf zukünftige Entwicklung hinsichtlich des Themas.
