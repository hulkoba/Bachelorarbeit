\chapter{\label{chap:grundlagen}Grundlagen}
Dieses Kapitel befasst sich mit den technischen und physikalischen Grundlagen.  
% -------------------------------------------------
% TECHNISCHE GRUNDLAGEN
% -------------------------------------------------
\section{\label{sec:technGrundlagen}Technische Grundlagen}
Im Rahmen dieser Arbeit wird eine \gls{Smartphone}-Anwendung erstellt, deren Implementierungsgrundlage die Software-Plattform Android und deren zentrales Merkmal die Bereitstellung von standortbezogenen Diensten ist. Die eigene Position wird mittels \gls{GPS} ermittelt, um in Verbindung mit der festen Position der nächsten \gls{LSA} die optimale Geschwindigkeit für das Erreichen der "'Grünen Welle"' zu errechnen. Im folgenden Abschnitt werden Funktionsweise und Besonderheiten der verwendeten Technologien beschrieben. 
% ANDROID
\subsection{Android}
Android ist ein von Google entwickeltes Linux-basiertes Betriebssystem für Mobilgeräte.\\ 
Android-Anwendungen werden mit der Programmiersprache Java und der Auszeichnungssprache \gls{XML} entwickelt. Mit dem Android \gls{SDK}\footnote{ Das Android \gls{SDK} steht unter \url{https://developer.android.com/sdk/index.html} zum Download bereit} werden die Werkzeuge und \glspl{API} zur Verfügung gestellt, die erforderlich sind, um Mobilanwendungen auf der Android-Plattform erzeugen zu können.\\ 
Zu den wichtigsten \gls{SDK}-Werkzeugen gehören der Android\gls{SDK} Manager, der AVD-Manager, der Emulator und der \gls{DDMS}. Der \gls{SDK} Manager verwaltet die \gls{SDK}-Pakete, sowie die installierten Pakete und System-Images. Der AVD-Manager bietet eine graphische Oberfläche, in der Android Virtuell Devices verwaltet und im Emulator ausgeführt werden können. Mithilfe des \gls{DDMS} können Android Anwendungen auf Fehler untersucht werden. \cite{android_sdk} \\\\
Mit dem Android \gls{NDK}\footnote{ Das Android \gls{NDK} steht unter \url{https://developer.android.com/tools/sdk/ndk/index.html} zum Download bereit} existiert auch eine Möglichkeit, mit dem Teile einer Anwendung in hardwarenahen Programmiersprachen wie C oder C++ implementiert werden können. Ein Programmcode, der in solchen Sprachen geschrieben ist, eignet sich zum Beispiel bei CPU-intensiven Operationen wie Signalverarbeitungen oder Physik-Simulationen besonders gut. Hier ist allerdings sicherzustellen, ob die erforderlichen Bibliotheken in dem \gls{SDK} auch verfügbar sind. \cite{android_ndk} \\
Einen Überblick über die komplexe Android-Systemarchitektur, welche nachfolgend (nach \cite{android} S. 2ff und \cite{androidVM} S.63ff) kurz beschrieben wird, zeigt die folgende Abbildung.\\
\begin{figure}[H]  
    \centering  
    \includegraphics[width=0.8\textwidth]{android_framework_details} 
    \grayRule
    \caption[Android-Systemarchitektur]{Die Android-Systemarchitektur Quelle: \cite{android_fig}}
    \label{fig:android}
\end{figure}
\paragraph{Linux Kernel: }
Android basiert auf dem Linux-Kernel. Wie in jedem Unix-System, stellt der Kerneltreiber für Hardware, Netzwerk, Dateisystemzugriff und Prozessmanagement bereit.
\paragraph{Android Runtime: }
Die Android Laufzeitumgebung nutzt die aus dem Apache-Harmony-Projekt nachimplementierten Java-Schnittstellen, die Dalvik \gls{VM} und die mit der Version 5.0 eingeführten Android Runtime (ART). \cite{android_5}\\
Der Großteil von Android ist in Java implementiert und wird von einer \gls{JVM} ausgeführt. Androids aktuelle Umsetzung der \gls{JVM} heißt Dalvik. Dalvik wurde speziell auf mobile Geräte zugeschnitten und kann den Java-Bytecode nicht direkt ausführen. So wird der kompilierte Java-Bytecode erneut in Dalvik-Bytecode kompiliert und von der Dalvik-\gls{VM}, oder ab der Version 5.0 von der ART ausgeführt.\\
Der Hauptunterschied der beiden \glspl{VM} besteht darin, dass ART die binäre Maschinensprache bei der \gls{App}-Installation, und nicht wie die Dalvik-\gls{VM}, bei der Ausführung erstellt.
\paragraph{\gls{HAL}: }
Der \gls{HAL} bietet die Möglichkeit eigene Gerätespezifikationen und deren benötigte Treiber, verbunden über eine Software-Schnittstelle zwischen dem Android-Stack und der Hardware, zu implementieren. \cite{android_hal}
\paragraph{Bibliotheken: }
Systemeigene Bibliotheken sind C/C++ Bibliotheken und sind vorinstalliert. Dazu gehören alle Bibliotheken im lila-Bereich von Abbildung \ref{fig:android}:
\begin{itemize}[leftmargin=0.7cm]
\renewcommand\labelitemi{--}
	\item \textsc{Surface Manager}: ~ Der für die Displayverwaltung verantwortliche Oberflächen-Manager
	\item \textsc{OpenGL/ES}: ~ Eine 2D und 3D -Grafikbibliothek
	\item \textsc{SGL}: ~ Eine 2D-Grafikbibliothek
 	\item \textsc{Media-Framework}: ~ Eine Medien-Bibliothek zur Wiedergabe von Audio- und Video-Daten 	
 	\item \textsc{FreeType}: ~ eine Bibliothek zur Darstellung von Computerschriften als Rastergrafik
 	\item \textsc{SSL}: ~ Das Secure-Socket-Layer für die Internet-Sicherheit
	\item \textsc{SQLite}: ~ Ist eine ausgereifte Datenbank die den internen Gerätespeicher nutzt 
	\item \textsc{Webkit}: ~ WebKit ist die Standard-Browser-Engine und erlaubt das Rendern und Anzeigen von HTML Seiten
	\item \textsc{libc}: ~ Eine C-Bibliothek mit Basisfunktion
\end{itemize}
\paragraph{Application Framework: }
Androids Application-Framework ist eine Umgebung die unterschiedliche Dienste zur Verfügung stellt. Sie bietet EntwicklerInnen Zugriff auf die im Kern verwendeten \glspl{API} sowie auf die Java-Bibliotheken die für Android erstellt wurden. 
\paragraph{Applications: }
Auf der obersten Ebene in Abbildung \ref{fig:android} befinden sich die Anwendungen die den täglichen Telefon-Bedarf wie Adressbuch, Media-Player, E-Mail, Internet-Browser etc. decken. Zusätzlich unterstützt Android verschiedene Anwendungen von Drittanbietern.
%
% MOBILE SENSING
%
\subsection{Mobile Sensorik} 
Ein Sensor\footnote{ aus dem Lateinischen, deutsch: "'\textit{fühlen}"'} ist ein Bauelement, das physikalische Eigenschaften wie Helligkeit, Temperatur oder Beschleunigung sowohl quantitativ als auch qualitativ erfassen und in ein analoges Signal umwandeln kann. \cite{sensor} \\\\
%\textit{Nahezu alle aktuellen \glspl{Smartphone} enthalten einen \gls{GPS}-Sensor, über denn der jeweilige Standort ermittelt wird. Im nächsten Abschnitt wird auch darauf eingegangen, wie Weiterhin können zur Ortsbestimmung Daten aus Funkzellen und naheliegender kabelloser Netzwerke hinzugezogen werden.}
% % MOBILE SENSORIK UNTER ANDROID %
%\subsubsection{Mobile Sensorik unter Android}
Die meisten Android-Mobilgeräte verfügen über integrierte Sensoren, die die Bewegung, Ausrichtung und verschiedene Umgebungsbedingungen messen. Diese Sensoren sind nützlich wenn man dreidimensionale Gerätbewegungen, Positionierungen oder Änderungen in der Umgebung des Gerätes überwachen möchte. So können zum Beispiel Spieleanwendungen den Beschleunigungssensor nutzen, um komplexe BenutzerInnengesten und Bewegungen wie Neigung, Erschütterung, Drehung oder Schwenkung erfassen.\\
Die Android-Plattform unterstützt Bewegungssensoren zum Messen von Beschleunigungen und Drehungen in drei Achsen, Umgebungssensoren zur Ermittlung verschiedener Umweltparameter wie Luftdruck und -feuchtigkeit oder Beleuchtung und Temperatur und Positionssensoren zum Messen der physikalischen Position des Gerätes. Android bietet mit dem Android Sensor Framework eine Sammlung von Klassen und Schnittstellen an mithilfe dessen man auf diese Sensoren zugreifen und deren Daten erfassen kann. \cite{android_sensor}  
%
% LOCATION BASED SERVICE
%
\subsection{Standortbezogene Dienste}
Standortbezogene Dienste (\gls{LBS}) sind Informationsdienste, die die geographische Position des Endgeräts nutzen, um für die NutzerInnen einen Mehrwert bereitzustellen. Sie ermöglichen den NutzerInnen, den eigenen Standort zu bestimmen, zu erfahren was sich in der Nähe befindet und dieses (z.B. in sozialen Netzwerken) zu bewerten. Für die Bereitstellung von \gls{LBS} sind die vier Schlüsselkomponenten Mobilgerät, Content-Provider, Kommunikationsnetzwerke und Lokalisierungstechniken erforderlich. Das Kommunikationsnetzwerk (z.B. Internet) übeträgt die Daten an den Content-Provider (z.B. Server), welcher die bearbeiteten Anfragen an das Mobilgerät sendet (vgl. \cite{gps} S. 4f).\\
Zur Bestimmung von Standortdaten gibt es mehrere Lokalisierungstechniken. Einige davon werden im Folgenden erklärt.
%
% GEOLOKATION NETWORK PROVIDER %
%
\subsubsection{Standortbestimmung via Netzwerk}
Netzwerkgestützte Standortbestimmung basiert auf Informationen des \glspl{WLAN} oder der Funkzellen des Mobilfunknetzes.\\ 
% MOBILFUNK %
Funkzellentriangulierung verwendet die bekannte Geschwindigkeit von Funksignalen, um den Abstand zum Empfangsgerät zu berechnen. Das Mobilgerät muss mit einem Funkturm in Verbindung stehen. Bewegt sich das Gerät, verbindet es sich mit einem anderen Turm und die Signalstärke des sich nähernden Turms wird stärker. Unter Kenntnis der einzigartigen ID und der Position des Funkturms mit dem das Gerät verbunden ist und ggf. derer, mit denen das Gerät zuletzt verbunden war, lässt sich der eigene Standort ermitteln (vgl. \cite{location} S. 8f). \\
Für eine gute Lokalisierung braucht es mindestens drei, vorzugsweise vier Mobilfunkmasten oder Antennen. In Städten, wenn sich mehr Signale der Mobilfunkmasten überlappen kann eine Genauigkeit von bis zu 200 Metern erreicht werden -- bei einer geringen Funkturmdichte sinkt diese auf bis zu mehreren Kilometern, wie in Abbildung \ref{fig:cell3} illustriert. \\
So eine Überlappung von drei Funktürmen ist in Abbildung \ref{fig:cell1} veranschaulicht. Diese Genauigkeit kann, wie in Abbildung \ref{fig:cell2} demonstriert, erhöht werden wenn Richtantennen auf dem Funkturm installiert sind. So kann zusätzlich die Richtung des Mobiltelefonsignals ermittelt werden (vgl. \cite{gps} S. 23). \\
\begin{figure}[H]
        \centering
        \begin{subfigure}[t]{0.23\textwidth}
                \includegraphics[width=\textwidth]{triangulation1}
                \caption{Überlappung von drei Funktürmen}
                \label{fig:cell1}
        \end{subfigure}
        ~ 
        \begin{subfigure}[t]{0.23\textwidth}
                \includegraphics[width=\textwidth]{triangulation2}
                \caption{Überlappung von zwei Funktürmen mit Richtantennen}
                \label{fig:cell2}
        \end{subfigure}
         ~ 
        \begin{subfigure}[t]{0.23\textwidth}
                \includegraphics[width=\textwidth]{triangulation3}
                \caption{Keine Überlappung von Funkturmsignalen}
                \label{fig:cell3}
        \end{subfigure}
        \grayRule
        \caption[Funkzellentriangulierung]{Funkzellentriangulierung Quelle: \cite{fig:cell}}
        \label{fig:cell}
\end{figure}
%Ähnlich der funkzellenbastierten Standorterkennung funktioniert die \gls{WLAN}-basierte.
\gls{WLAN}-basierte Standorterkennung funktioniert durch Funksignale, die von \gls{WLAN} -Access Points\footnote{ Drahtloser Zugangspunkt, der als Schnittstelle für kabellose Kommunikationsgeräte dient} ausgesendet werden, um den genauen Standort eines jeden Wi-Fi-fähigen Gerätes zu ermitteln. Bei Aktivierung scannt die Software die Umgebung nach \gls{WLAN}-Access Points und berechnet die Position des Gerätes indem sie die empfangenen Signale mit der Referenzdatenbank vergleicht. Wie bei der funkzellenbastierten Standorterkennung erhöht sich auch hier die Genauigkeit (auf 20 bis 40 Meter in Europa) mit wachsender Signaldichte. \\
Effektiv werden die gleichen Prinzipien der Funkzellentriangulation übernommen. Nur werden WiFi-SSIDs\footnote{ Service Set Identifer (SSID), entspricht der MAC-Adresse des Wireless Access Points oder wird als Zufallszahl erzeugt} statt Funkzellenidentifikationsnummern zum Feststellen der Sendequellen verwendet (vgl. \cite{gps} S. 32ff). \\
Netzwerkgestützte Standortbestimmung schont zwar im Gegensatz zu \gls{GPS} den Akku und ist innerhalb von Gebäuden nutzbar, liefert jedoch wesentlich ungenauere Ergebnisse.
% GEOLIKATION GPS %
\subsubsection{Standortbestimmung via \gls{GPS}}
Die satellitengestützte Positionsbestimmung \gls{GPS} gewährleistet die Bestimmung des exakten Standpunktes und ist so wesentlicher Bestandteil ortsgebundener Anwendungen wie zum Beispiel der in Kapitel \ref{chap:state} beschriebenen. \\
\gls{GPS} wurde ursprünglich vom US-Militär entwickelt und dann Mitte der 90er Jahre der Öffentlichkeit zur Verfügung gestellt. Noch heute bleibt es mit einer Genauigkeit von bis zu drei Metern die genaueste Lokalisierungstechnologie (vgl. \cite{gps} S. 24f).\\
Das Globale Positionsbestimmungssystem umfasst 27 Satelliten, die ständig die Erde umkreisen und dabei kontinuierlich ihre eigene, aktuelle Position und Almanach-Daten aussenden. Almanach-Daten enthalten Daten über jeden Satelliten in der Konstellation. Abbildung \ref{fig:gps} zeigt eine Darstellung der GPS-Satellitenkonstellation. Jeder Satellit folgt einer definierten Bahn, sodass mindestens vier Satelliten von jedem Standpunkt von der Erde aus "'sichtbar"' sind (vgl. \cite{location} S. 4f).\\
Mittels Triangulation errechnet daraus ein \gls{GPS}-Empfänger die eigene Position. Daten von einem einzigen Satelliten lässt die Position des \gls{GPS}-Empfängers auf einen großen Bereich der Erde einschränken. Das Hinzufügen eines zweiten engt die Position auf den Bereich, in dem sich die zwei Bereiche überlappen ein. Mit den Daten eines dritten Satelliten bekommt man bereits eine relativ genaue Positionierung. Mit jedem weiteren Satelliten wird die Präzision erhöht und durch zusätzliche Positionsinformation wie Höhe erweitert. \gls{GPS}-Empfänger benutzen regelmäßig vier bis sieben, oder gar mehr Satelliten (vgl. \cite{gps} S. 25f).\\
\begin{figure}[H]  
    \centering  
    \includegraphics[width=0.4\textwidth]{gps} 
    \grayRule
    \caption[GPS Satelliten Konstellation]{GPS Satelliten Konstellation  Quelle: \cite{fig:gps}}
    \label{fig:gps}
\end{figure}
Trotzdem hat \gls{GPS}, insbesondere für mobile Plattformen einige Nachteile. Es verbraucht viel Energie, was die Akkulaufzeit des Mobiltelefons beeinträchtigt. Bevor der Standort berechnet werden kann, müssen mehrere Satelliten ermittelt werden, was nach dem Kaltstart\footnote{ Start ohne aktuelle Satellitendaten} einige Zeit in Anspruch nehmen kann (vgl. \cite{location} S. 5).\\\\
Inzwischen wird auf etlichen Mobiltelefonen zusätzlich das russische \gls{GNSS} names GLONASS\footnote{ GLONASS, russisch Globalnaja nawigazionnaja sputnikowaja sistema, dt. Globales Satellitennavigationssystem} eingesetzt\footnote{ \url{http://en.wikipedia.org/wiki/List_of_smartphones_supporting_GLONASS_navigation}}.
Die europäische Variante in der Testphase mit ähnlichem Aufbau heißt Galileo, das chinesische (ausschließlich in China nutzbar) \gls{GNSS} heißt BeiDou\footnote{ BeiDou, dt. Großer Bär} (vgl. \cite{gps} S. 24).
% A-GPS
\subsubsection{\gls{A-GPS}}
Viele moderne \glspl{Smartphone} sind heutzutage mit der \gls{GPS}-Variante \gls{A-GPS} ausgestattet. Hierbei werden Zusatzinformationen über die nächstgelegenen Satelliten via Mobilfunk bezogen, sodass die Erstbestimmung der Position sehr viel schneller ablaufen kann.
Voraussetzung hierfür ist die Ausstattung der Mobilfunktürme mit \gls{GPS}-Empfängern, welche kontinuierlich die Satellitenpositionen beziehen. Diese Daten werden, sobald angefordert, an das Mobiltelefon gesendet (vgl. \cite{gps} S. 26f). \\
\gls{A-GPS} benötigt daher nur die Sicht auf einen Satelliten und erzielt durch die Einbeziehung der Assistenzinformationen trotzdem genauere Ergebnisse in der Ortsbestimmung.
Dieses ist insbesondere in Städten (Einschränkung des \gls{GPS} durch z.B. hohe Gebäude) von Vorteil.
% GEOLOLATION UNTER ANDROID %
\subsubsection{Standortbestimmung unter Android}
Android unterstützt mit dem \texttt{android.location} Paket den Zugriff auf die Ortungsdienste. Als zentrale Komponente des Location Frameworks stellt der \texttt{LocationManager} \glspl{API} zur Lokalisierung des Geräts bereit. Mit dem \texttt{LocationManager} ist die Anwendung in der Lage alle Location Provider\footnote{ Location Provider, dt. Standortanbieter. Ein Standortanbieter bietet regelmäßige Berichte über die geographische Lage des Gerätes} des letzten bekannten Standortes abzufragen, sich für regelmäßige Updates zur Position des Gerätes anzumelden und sich wieder abzumelden wenn sich das Gerät außerhalb gegebener Parameter befindet. \cite{android_gps} \\
Die geographische Positionsangabe besteht aus Längengrad (Latitude) und Breitengrad (Longitude). Beide werden unter anderem vom \texttt{LocationManager} über das \texttt{Location}-Objekt als Gleitkommawert geliefert. Daneben auch Informationen wie die Höhe in Metern über der Meereshöhe, Peilung, Zeitstempel und die Geschwindigkeit.\\
Der Abstand zwischen zwei Punkten kann mit \texttt{Location.distanceBetween()} abgerufen werden. Diese Methode bekommt als Parameter die Längen- und Breitenkoordinaten der beiden Punkte und ein Array mit \texttt{float}-Werten für die Ergebnisse übergeben. Dieses Array muss eine Größe von mindestens eins haben und gibt die ungefähre Entfernung in Metern im Index Null zurück. Abstandsberechnungen werden mit dem Referenzsystem WGS84\footnote{ World Geodetic System}-Ellipsoid definiert ( vgl. \cite{location} S.43). 
\\\\
Der Fused Location Provider verwaltet die zugrunde liegenden Ortungstechnologie. Er ermöglicht zum Beispiel den Zugriff auf letzte Position und minimiert den Energieverbrauch der Anwendung indem er auf Grundlage aller eingehenden Standortanfragen und verfügbaren Sensoren die effizientesten auswählt. \cite{android_fused}
\clearpage
% -------------------------------------------------
% MATHEMATISCHE GRUNDLAGEN
% -------------------------------------------------
\section{\label{sec:mathGrundlagen}Berechnung der Geschwindigkeitsempfehlung}
Präsentiert das System während der Anwendung eine Geschwindigkeitsempfehlung, ist diese abhängig von der Fahrgeschwindigkeit und vom Abstand zur Ampel. Für einen Streckenabschnitt $s$ zwischen zwei Punkten (Position der Ampel und Position der Rades) wird die Zeitspanne $\Delta t$ benötigt. Diese errechnet man mit $t_{2} - t_{1}$. In Kenntnis dieser beiden Größen lässt sich dann die erforderliche Durchschnittsgeschwindigkeit im untersuchten Streckenabschnitt errechnen.\\\\ 
Angenommen die Progressionsgeschwindigkeit $v$ wird zum Zeitpunkt $t_{1}$ ermittelt, die \gls {LSA} schaltet zum Zeitpunt $t_{2}$ auf Rot und Abstand zur Ampel beträgt $s$, dann gilt: \\
\[ v = \frac{s}{t_{2} - t_{1}} \] \\
Die aus den erfassten Daten erstellten Ampelsignalpläne und Position der angesteuerten Ampel sind die Basis dieser Berechnung. Die aktuelle Position des Fahrrads wird vom \gls{GPS}-Sensor des \glspl{Smartphone} ermittelt und daraus der Abstand zur Ampel errechnet. Die Abbildung \ref{fig:vst} illustriert die Grundlagen der Geschwindigkeitsberechnung: \\
\begin{figure}[H]  
    \centering  
    \includegraphics[width=\textwidth]{vst}  
    \grayRule
    \caption[Berechnung Progressionsgeschwindigkeit]{Visualisierung der Geschwindigkeitsberechnung}
    \label{fig:vst}
\end{figure}
\clearpage
Da das System für Fahrräder entwickelt wird, ist auch die maximal mögliche Beschleunigung von Bedeutung. FahrradfahrerInnen können nicht unbegrenzt beschleunigen, um die gewünschte Geschwindigkeit zu erreichen. Sie ist abhängig von der Geschwindigkeit und der Zeitspanne, in der diese zu erreichen ist. Die Formel für Beschleunigung lautet:
\[ a = \frac{v}{(t_{2} - t_{1})^{2}} \]\\
 Um die ensprechende \gls{LSA} während der Grünphase zu passieren, muss letztendlich die empfohlene Geschwindigkeit $v$ eingehalten werden, wobei die maximal mögliche Beschleunigung $a$ nicht überschritten werden kann.
