 \textit{Der Beschleunigungssensor ist ein Hardwaresensor, der dazu benutzt wird, Position, Bewegung, Neigung, Erschütterung, Vibration und natürlich Beschleunigung des Gerätes zu messen.Es gibt bis zu 3-Achsen Beschleunigungssensoren, die meistens zum Erkennen der Ausrichtung des \glspl{Smartphone} genutzt werden und somit das Display beim Anschauen von Bildern, Webbrowsern oder Musikplayern in die passende Richtung vom Portrait-Modus (senkrecht) zum Landscape-Modus (waagrecht) zu drehen. In Kombination mit \gls{GPS} kann das \gls{Smartphone} dank ihm sogar erkennen, welche Art Transportmittel (Fahrrad, Bus, U-Bahn) der Nutzer gerade benutzt und bestimmte Muster wie z.B. Rennen, Gehen oder Stehen unterscheiden.\\
\gls{GPS} erlaubt dem \gls{Smartphone} sich selber zu lokalisieren und den exakten Standpunkt auf der Erde zu bestimmen. Es hilft locationbased\footnote{ ortsgebunden} \Glspl{App} wie z.B Navigation, lokale Suche nach Shops, Restaurants etc. oder soziale Netzwerke wie Facebook oder Foursquare nötige Informationen zu ermitteln. Der Kompass erweitert die Möglichkeiten der Lokalisationsermittlung eines \gls{Smartphone}s. Er bestimmt den Winkel des Geräts relativ zum Nordpol der Erde. Der Kompass besitzt einen Magnet, der mit dem magnetischen Feld der Erde interagiert und sich entsprechend zu einem der Pole ausrichtet. Zusammen mit dem Gyroskop Sensor verbessern \gls{GPS} und Kompass die Präzision von locationbased Applikationen.Der Gyroskop Sensor bestimmt die Rotations- und Drehgeschwindigkeit des \gls{Smartphone}s auf seinen drei Achsen gegenüber dem Weltkoordinatensystem.}
