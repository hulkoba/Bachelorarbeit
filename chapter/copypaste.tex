Javascript - vormals noch als Livescript bekannt -  ist eine dynamische Skriptsprache, die 1995 von der Firma Netscape als Teil ihres Browsers entwickelt wurde. Das damalige Implementierungsziel war es, das Web um weitere Funktionalitäten zu erweitern und zu dynamisieren. Um die Anzahl der Server-Anfragen zur Validierung von Eingabefelder zu reduzieren, gehörte vor allem die clientseitige Validierung zu der gewünschten Funktionalität.
Javascript zeichnet sich vor allem durch den prototyp-basierenden Ansatz aus. Dies bedeutet, dass ein konstruiertes Objekt immer die Eigenschaften seines Prototypen erbt. Weitere Besonderheit dieser Sprache ist die dynamische Typisierung. Diese Charaktereigenschaft beschreibt die Tatsache, dass Variablen keine Typdeklaration benötigen. Heute ist Javascript eine kaum wegzudenkende Komponente des Internets und kann seither eine Fülle an Funktionalität vorweisen. Man unterscheidet hierbei zwischen der client- und der serverseitigen Verwendung von Javascript.\\
Clientseitiges Javascript findet in heutigen Internetbrowser sein Zuhause. Elementare Funktion von heutigen Webanwendungen ist es, das Document Object Model (DOM), also die baumartige Struktur der Webseite, zu manipulieren. Durch Zugriff auf die globale \textit{document} Variable kann der Entwickler das Dokument und seine einzelne Elemente ansprechen und bearbeiten. Der Zugriff auf das BOM, also das Browser Object Model, ermöglicht dem Anwender auch Manipulationen außerhalb des \textit{document} Kontextes. So ist dem Entwickler gestattet das Browserfenster in seiner Größe und in seinem Status zu  bearbeiten. Ebenfalls kaum noch wegzudenkende Funktionalität ist das Nachladen von Informationen ohne die Webseite neu laden zu müssen. Unter Verwendung des \textit{XMLHttpRequest} Objektes ist es möglich, Netzwerkzugriffe zu initialisieren und zu steuern. Die Gesamtheit all dieser Funktionen bildet die Basis heutiger Webanwendungen. \\
Im  Abschnitt \ref{Node.js} soll die serverseitige Verwendung von Javascript am Beispiel von NodeJS beschrieben werden. 
