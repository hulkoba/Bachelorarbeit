\chapter{\label{chap:implementierung}Der Prototyp}
In diesem Kaptitel wird die nach dem in Kapitel \ref{chap:entwurf} präsentierten Lösungswegs die detaillierte Beschreibung der technischen Realisierung der Anwendung vorgestellt.\\
Nach der Erklärung der Konfigurationsdateien wird auf die Umsetzung der Szenarien eingegangen. Im Zuge dessen werden die implementierten Algorithmen vorgestellt, wobei sich der erste mit dem Auffinden der nächsten relevanten Ampel befasst und der zweite die empfohlene Geschwindigkeit berechnet.
\section{Die Manifest- und build.gradle-Datei}
Das Android-Manifest dient der Festlegung wichtiger Eigenschaften der Anwendung und gehört zu jedem Android-Projekt. (+ gradle...)\\

Die \gls{XML}-Datei (\texttt{AndroidManifest.xml}) ist im Hauptverzeichnis des Projekts zu finden und ist im Listing \ref{lst:manifest} abgebildet. \\
In der zweiten Zeile wird hier der Paketname des Programms festgelegt. 
Im \texttt{application}-Tag werden Variablen gesetzt, die das in dargestellte Icon und den Namen der Anwendung definieren. Darüber hinaus wird hier die \gls{Activity} der Applikation definiert. Zuerst wird der Name der \gls{Activity} gesetzt. Die Variable \texttt{screenOrientation} legt das Format der Anzeige fest und verhindert ein automatisches Drehen des Bildschirms. Im \texttt{intent-filter}-Tag dass diese Activity beim Start der App ausgeführt wird. Hätte die Anwendung über mehrere \glspl{Activity} implementiert, würden die anderen ebenfalls hier aufgeführt werden.\\
Unterhalb des \texttt{application}-Tags, in Zeile 17, werden nun die Berechtigung des \gls{GPS}-Zugriffs der Applikation, um Standortdaten, also die jeweiligen geographischen Kordinaten des Endgeräts zu beziehen gesetzt.
\begin{center}
\rule{35em}{0.5pt} \lstinputlisting[language=XML, firstline=2, lastline=21, caption={AndroidManifest.xml}, label=lst:manifest]{code/manifest.xml}
 \rule{35em}{0.5pt}
\end{center}
Für welche Android Versionen die Anwendung geschrieben wurde (\texttt{targetSdkVersion}) und das minimale \gls{API}-Level der Anwendung, also unter welcher Version die App noch ausgeführt werden kann:\\
\begin{center}
\rule{35em}{0.5pt} \lstinputlisting[language=JSON, firstline=6, lastline=14,  caption={build.gradle}, label=lst:gradle]{code/build.gradle} \rule{35em}{0.5pt}
\end{center}
Hier wird der Name der \gls{Activity}-Klasse gesetzt und im \texttt{intent-filter}-Tag festgelegt, dass diese \gls{Activity} als \texttt{MainActivity} beim Start der Anwendung ausgeführt wird. \textit{label:android:?}
\section{MainActivity-Klasse}
\section{Umsetzung Szenarien}
\subsection{?Einlesen der Ampeldaten?}
\subsection{Ermittlung der nächsten Ampel}
\subsection{Algorithmus für die Geschwindigkeitsempfehlung}

%
% TEST
%
\chapter{Evaluierung}
\section{Systemtest}
ist das GPS schnell/genau genug fürs Radfahren?
Optimierung ggf. umsetzen\\
Personenbezogene Daten aufnehmen? Höchstgeschwindigkeit, maximale Beschleunigung
