\documentclass[a4paper,
oneside,
11pt, 
headsepline,
abstracton,
]{scrreprt} %%scrreprt
\usepackage[T1]{fontenc}
\usepackage[utf8]{inputenc} % Umlaute für linux
\usepackage[english, ngerman]{babel}
\usepackage{lmodern}	 %Type1-Schriftart fuer nicht-englische Texte
\usepackage{listings} %dient der Darstellung von Quellcode
\usepackage{graphicx}
\usepackage{wrapfig}
\usepackage[justification=raggedright, format=hang,labelsep=quad]{caption}
\usepackage{subcaption}
\usepackage{titleref}
% aller Bilder werden im Unterverzeichnis figures gesucht:
\graphicspath{{img/}}
\usepackage{cite}
% Paket für Quotes
\usepackage[babel, german=swiss]{csquotes}
\usepackage{txfonts} %Schriftart Times New Roman
\usepackage{titlesec} % formatiert titel-absätze
\usepackage{booktabs}
\usepackage[table,xcdraw,dvipsnames ]{xcolor}
\usepackage{float}
\usepackage[left=30mm,right=25mm,top=25mm,bottom=30mm]{geometry} %Seitenränder
\usepackage[pdfborder={ 0 0 0 }, breaklinks]{hyperref}
\usepackage{enumitem}
\usepackage{xspace} % setzt Leerzeichen, wenn welche hingehören 
\usepackage{scrhack} %The package scrhack redefines macros of other packages. It should be loaded at the very end
\linespread {1.25}\selectfont
\usepackage[ngerman]{translator}
\usepackage[
nonumberlist,   %keine Seitenzahlen anzeigen
%nopostdot,      %keine Punkte am Ende
acronym,       %ein Abkürzungsverzeichnis erstellen
toc]            %Einträge im Inhaltsverzeichnis
{glossaries}
\makeglossaries
%Glossar-Befehle anschalten
\newglossaryentry{C2X}{
    name=Car-to-X-Kommunikation, 
    description={ direkter Informationsaustausch zwischen Fahrzeugen jeglicher Art, Verkehrsleittechnik wie z.B. Lichtsignalanlagen und Verkehrsleitzentralen}
}
\newacronym{App}{App}{Applikation}
\newacronym{WLAN}{WLAN}{Wireless Local Area Network}
\newacronym{LSA}{LSA}{Lichtsignalanlage}
\newacronym{OSM}{OSM}{Open Street Map}
\newacronym{BMW}{BMW}{Bayerische Motoren Werke}
\newacronym{TUM}{TUM}{Technische Universität München}
\newacronym{VW}{VW}{Volkswagen}
\newacronym{LED}{LED}{Licht-emittierende Diode, auch Lumineszenz-Diode || Leuchtdiode}
%\newacronym{C2X}{C2X}{Car-to-X oder Vehicle-to-X}
\newacronym{I2C}{I2C}{Infrastructure-to-Car oder Infrastructure-to-Vehicle}
\newacronym{C2R}{C2R}{Car-to-Roadside oder Vehicle-to-Roadside}
\newacronym{C2I}{C2I}{Car-to-Infrastructure oder Vehicle-to-Infrastructure}
\newacronym{DSRC}{DSRC}{Dedicated Short Range Communication}
\newacronym{MIT}{MIT}{Massachusetts Institute of Technology}
\newacronym{SQL}{SQL}{Structured Query Language}
\newacronym{GPS}{GPS}{Global Positioning System}
\newacronym{UI}{UI}{User Interface}
\newacronym{GUI}{GUI}{Graphical User Interface}
\newacronym{HTML}{HTML}{HyperText Markup Language}
\newacronym{CSS}{CSS}{Cascading Style Sheet}
\newacronym{XML}{XML}{Extensible Markup Language}

\addtokomafont{chapter}{ \scshape \mdseries}
\addtokomafont{section}{\huge \scshape \mdseries \color{darkgray}} 
\addtokomafont{subsection}{\LARGE \scshape \mdseries \color{gray}}
\addtokomafont{subsubsection}{\large \scshape \mdseries}
\setlength{\parindent}{0pt}  % keine Einrückungen nach Absätzen

%------------------------------------------------------------------------
% CODE
%------------------------------------------------------------------------
\definecolor{lightgray}{gray}{0.95}
\lstdefinelanguage{XML}
{
  morestring=[b]",
  morecomment=[s]{<?}{?>}{>}{<},
  stringstyle=\color{purple},
  identifierstyle=\color{violet},
  commentstyle=\color{OliveGreen}\ttfamily,
  morekeywords={xmlns,version,type}% list your attributes here
}
\lstdefinelanguage{JSON}{
	morestring=[b]",
	stringstyle=\color{purple},
    literate=
     *{0}{{{\color{violet}0}}}{1}
      {1}{{{\color{violet}1}}}{1}
      {2}{{{\color{violet}2}}}{1}
      {3}{{{\color{violet}3}}}{1}
      {4}{{{\color{violet}4}}}{1}
      {5}{{{\color{violet}5}}}{1}
      {6}{{{\color{violet}6}}}{1}
      {7}{{{\color{violet}7}}}{1}
      {8}{{{\color{violet}8}}}{1}
      {9}{{{\color{violet}9}}}{1}
      {false}{{{\color{violet}false}}}{1}      
      {true}{{{\color{violet}true}}}{1}
      {:}{{{\color{blue}{:}}}}{1}
      {,}{{{\color{blue}{,}}}}{1}
      {\{}{{{\color{blue}{\{}}}}{1}
      {\}}{{{\color{blue}{\}}}}}{1}
      {[}{{{\color{blue}{[}}}}{1}
      {]}{{{\color{blue}{]}}}}{1},     ,
}
\lstset{
   captionpos=b,
   basicstyle=\scriptsize\ttfamily,
   keywordstyle=\bfseries\ttfamily\color{violet},
   stringstyle=\color{purple}\ttfamily,
   commentstyle=\color{OliveGreen}\ttfamily,
   backgroundcolor=\color{lightgray},
   emph={square}, 
   emphstyle=\color{blue}\texttt,
   emph={[2]root,base},
   emphstyle={[2]\color{yac}\texttt},
   showstringspaces=false,
   flexiblecolumns=false,
   tabsize=4,
   numbers=left,
   numberstyle=\tiny,
   numberblanklines=false,
   stepnumber=1,
   numbersep=5pt,
   xleftmargin=15pt
 }

%------------------------------------------------------------------------
% DOKUMENT
%------------------------------------------------------------------------
\begin{document}
\setlist{noitemsep}  %% verringert Zeilenabstand bei aufzaehlungen
\pagestyle{empty} %%Keine Kopf-/Fusszeilen auf den ersten Seiten.
% 
%Deckblatt
%
\begin{titlepage}

%\newcommand{\HRule}{\rule{\linewidth}{0.5mm}} % Defines a new command for the horizontal lines, change thickness here

\center % Center everything on the page
 
%----------------------------------------------------------------------------------------
%	HEADING SECTIONS
%----------------------------------------------------------------------------------------

\begin{center}
\includegraphics[width=0.8\textwidth]{img/beuth}  \\[2cm]
\end{center}

\begin{Huge}
Bachelorarbeit
\end{Huge}\\[0.5cm]
\LARGE{Medieninformatik}\\
\large{Fachbereich 6}\\[0.5cm]
%----------------------------------------------------------------------------------------
%	TITLE SECTION
%----------------------------------------------------------------------------------------

\rule{\textwidth}{1.6pt}\vspace*{-\baselineskip}\vspace*{2pt} % Thick horizontal line
\rule{\textwidth}{0.4pt}\\[\baselineskip] % Thin horizontal line
{\LARGE Ampelphasen-Informationssystem für FahrradfahrerInnen\\ auf Grundlage persistenter geo- und zeitbasierter Daten}\\[\baselineskip]

\rule{\textwidth}{0.4pt}\vspace*{-\baselineskip}\vspace{3.2pt} % Thin horizontal line
\rule{\textwidth}{1.6pt}\\[\baselineskip]\vspace{3.2pt} % Thick horizontal line

{\Large Berlin, den \today{}} \\[7cm]
%----------------------------------------------------------------------------------------
%	AUTHOR SECTION
%----------------------------------------------------------------------------------------

\begin{minipage}{0.4\textwidth}
\begin{flushleft} \large
\emph{Autorin:}\\
Jacoba \textsc{Brandner} \\
\emph{Matrikelnummer:}\\
786635
\end{flushleft}
\end{minipage}
~
\begin{minipage}{0.45\textwidth}
\begin{flushright} \large
\emph{Betreuerin:} \\
Frau Prof. Dr. Gudrun \textsc{Görlitz} \\
\emph{Gutachterin:} \\
Frau Prof. Dr. Petra \textsc{Sauer} \\
\end{flushright}
\end{minipage}\\


%----------------------------------------------------------------------------------------
%	LOGO SECTION
%----------------------------------------------------------------------------------------

% Include a department/university logo - this will require the graphicx package
%\includegraphics[scale=0.1]{extra/parolu} 

%----------------------------------------------------------------------------------------

\vfill % Fill the rest of the page with whitespace

\end{titlepage}
 
%
%Inhaltsverzeichnis
%
%\setcounter{secnumdepth}{3} %\setcounter{tocdepth}{3}
\renewcommand{\contentsname}{Inhalt} %Inhaltsverzeichnistitel = Inhalt
\tableofcontents 		

\cleardoublepage 	
\pagestyle{headings}	 	%%Ab hier die Kopf-/Fusszeilen: headings / fancy .
%\hyphenation{JavaScript} %trennt JavaScript nie
%
% Kapitel
\begingroup
\let\titlepage\par
\let\endtitlepage
\let
\selectlanguage{ngerman}
\begin{abstract}
Im Berliner Verkehrswesen ist ein deutlicher Trend zu bemerken. Das Fahrrad wird zum ökologischen und gesundheitlichen, aktiven Lebensstil und wird dem hohen Verkehrsaufkommen der Automobile, insbesondere in der Stadtregion, entgegenwirken. “Fahrradfahren boomt in Berlin stärker als bislang bekannt”  (J.Anker, Berliner Morgenpost, am 6.06.2014)\\
Neue Fahrradwege und Vergrößerung des Fahrradstraßennetzes sind regionale Baumaßnahmen, die dabei aktuell diesen Fahrradtrend unterstützen. \\
Grund der neuen Fahrradeuphorie ist nicht zuletzt die erfolgreiche Etablierung der E-Bikes. E-Bikes erfreuen sich großer Beliebtheit und ermöglichen auch längere Touren ohne große Anstrengung.
\end{abstract}
\selectlanguage{english}
\begin{abstract}
Im Berliner Verkehrswesen ist ein deutlicher Trend zu bemerken. Das Fahrrad wird zum ökologischen und gesundheitlichen, aktiven Lebensstil und wird dem hohen Verkehrsaufkommen der Automobile, insbesondere in der Stadtregion, entgegenwirken. “Fahrradfahren boomt in Berlin stärker als bislang bekannt”  (J.Anker, Berliner Morgenpost, am 6.06.2014)\\
Neue Fahrradwege und Vergrößerung des Fahrradstraßennetzes sind regionale Baumaßnahmen, die dabei aktuell diesen Fahrradtrend unterstützen. \\
Grund der neuen Fahrradeuphorie ist nicht zuletzt die erfolgreiche Etablierung der E-Bikes. E-Bikes erfreuen sich großer Beliebtheit und ermöglichen auch längere Touren ohne große Anstrengung.
\end{abstract}
\endgroup

\chapter{\label{chap:einleitung}Einführung}
\section{Motivation}
Im Berliner Verkehrswesen ist ein deutlicher Trend zu bemerken. Das Fahrrad wird zum ökologischen und gesundheitlichen, aktiven Lebensstil und wird dem hohen Verkehrsaufkommen der Automobile, insbesondere in der Stadtregion, entgegenwirken. “Fahrradfahren boomt in Berlin stärker als bislang bekannt”\cite{Mopo}\\\\
%Neue Fahrradwege und Vergrößerung des Fahrradstraßennetzes sind regionale Baumaßnahmen, die dabei aktuell diesen Fahrradtrend bekräftigen. \cite{Mopo} \\
Grund der neuen Fahrradeuphorie ist nicht zuletzt die erfolgreiche Etablierung der E-Bikes\footnote{ Elektrofahrrad. Ein Fahrrad mit elektrischem Hilfsmotor}. E-Bikes erfreuen sich großer Beliebtheit und ermöglichen auch längere Touren ohne große Anstrengung. (Vgl. \cite{ebikes} S.70ff)\\ 
Die Digitalisierung der Autoinnenräume mit Navigation und Bordelektronik sowie die Verbindungen zu \glspl{Smartphone} stellen aktuell keine Besonderheit mehr dar. Wird das Fahrrad nun als „vollwertiges“ Mitglied im Straßenverkehr angesehen, kann zusätzliche Elektronik wie Navigation die FahrradfahrerInnen unterstützen. Sicherheit und eine rechtzeitige Ankunft am Ziel sind die Hauptaspekte der VerkehrsteilnemerInnen. Das Halten an der Ampel kann dabei schnell zu Verzögerungen führen. Doch wer die Restzeit im Voraus kennt, kann sich darauf einstellen und so die verlorenen Zeitabschnitte reduzieren.
% ### ZIELSTELLUNG ###
\section{Zielstellung}
Der Fahrtfluss der RadfahrerInnen soll nicht unnötig unterbrochen werden. Rote Ampeln zwingen zum Anhalten -- das Anfahren kostet Kraft und ist deshalb unbeliebt. So kommt es, dass viele RadfahrerInnen die Straße bei rot überqueren und die Verkehrssicherheit aller gefährden, wo doch das das Radfahren an sich zur Gesundheit beiträgt und gut für die Umwelt ist. Angesichts des Nutzenpotentials eines Ampelinformationssystems lässt sich die Zielstellung klar und deutlich formulieren. Durch reibungsloses Passieren der Ampeln wird der Verkehr sicherer und das Radfahren attraktiver.\\
Um die Ampeldaten zu erfassen, gibt es verschiedene Möglichkeiten. Eine 100 prozentige Deckung erreicht man nicht einmal durch manuelle Ablesung jeder Ampel, denn viele \acrlongpl{LSA} werden verkehrsabhängig gesteuert. Fußgängerampeln beeinflussen erst nach Knopfdruck den Verkehr, Funkempfänger oder Infrarotdetektoren in Straßenbahnen oder Bussen wird durch Induktionsschleifen in der Fahrbahn der Verkehr erfasst und angepasst. Weiter sind Busse und Straßenbahnen in Berlin in der Lage, aus gewisser Entfernung über Funk Grün anzufordern, was ebenfalls in den Verkehr eingreift. ( Vgl. \cite{lsa_bln} S.4f) \\
%\textit{Die Berliner Verkehrslenkungszentrale stellt für diese Arbeit verkehrstechnische Unterlagen wie einen Lageplan, Signalzeitenpläne und Daten der verkehrsabhängigen Steuerung von \acrlongpl{LSA} von fünf Kreuzungen auf gewünschter Strecke als Basis für den zu entwickelnden Prototypen zur Verfügung.} 
Absolut entscheidend ist die Richtigkeit der Datengrundlage, also die Position und Signalschaltpläne der Ampeln, denn darauf baut die Funktionalität der Anwendung auf.
Für die Auswertung dieser Daten werden die potentiellen Wartezeiten an der nächsten Ampel vorzeitig errechnet und den FahrerInnen mitgeteilt. Resultierend können die NutzerInnen die Geschwindigkeit anpassen und die verbleibende Wegstrecke zur Ampel nutzen, um bei Grün ohne anzuhalten die Kreuzung zu überqueren. Für die Datenerhebung werden zugleich die mobilen Systeme der RadfahrerInnen genutzt. So kann zunächst der Prototyp des Ampelhinweissystems, beispielhaft für die Stadt Plau am See, entwickelt werden.\\\\
Das Ziel der Arbeit ist ein Konzept und dessen prototypische Anwendung eines Ampelhinweissystem, welches einem auf Basis der zu erfassenden Datengrundlage Informationen über die Ampelschaltung zukommen lässt und die NutzerInnen so interaktiv durch das Verkehrsnetz führt.
%
% AUFBAU DER ARBEIT
%
%\section{Aufbau der Arbeit}

\chapter{State of the art}
Die Verkehrsstrategie des Senats sieht vor, dass das Radfahren bis zum Jahr 2025 20 Prozent des Gesamtverkehrs ausmachen soll. \cite{Mopo}
"Wir brauchen eine intelligente Konstruktion, die alle Verkehrsarten verbindet", sagte Stadtentwicklungssenator Michael Müller (SPD).\\
Sowohl statisch an Radwegen, als auch für den Einsatz in Kraftfahrzeugen gibt es bereits Projekte zu Ampelassistenten in Bordcomputern, Navigationssystemen, oder aber auch als App die rote Ampeln erkennen und die optimale Fahrtgeschwindigkeit für die Grüne Welle ermitteln.\\\\
\gls{C2X}- Kommunikation oder/und \gls{GLOSA} definieren?\\
auch fahrraderweiterungssysteme aufführen? Helios-lenker, Cobi app etc?
% ### Auto ###
\section{Bestehende Konzepte}
Unter dem Prinzip "'Grüne Welle"' wird die Abstimmung der Ampelschaltzustände, sodass ein Fahrzeug in einer bestimmten Geschwindigkeit mehrere Ampeln passieren kann ohne anzuhalten, verstanden. Der folgende Abschnitt soll die existierenden Lösungen und Ansätze für die Ampelinformationssystemen darstellen.
\subsection{Grüne Welle auf Radwegen}
In Kopenhagen unterstützen grüne \gls{LED}s auf Radwegen die Radfahrer indem sie wenn diese mit einem Tempo von 20 km/h fahren, sie begleiten und so signalisieren, dass sie sich auf der Grünen Welle befinden. 
\begin{figure}[H]  
    \centering  
    \includegraphics[width=0.7\textwidth]{copenhagen}
    \label{fig:copenhagen}
    \caption[Grüne Welle durch \gls{LED}s]{\gls{LED}s signalisieren die Grüne Welle bei 20 km/h  Quelle: \cite{NYT}}
\end{figure}
Zusätzlich erkennen Sensoren im Radweg Fahrradgruppen und veranlassen dann die Ampel zu einer längeren Grünphase. In einem anderen Stadtteil sind Leuchttafeln, die die verbleibende Zeit der Ampelphase anzeigen, am Radwegrand installiert\footnote{\cite{KopIng}}.\\
Mit Kopenhagen als Vorbild hat Berlin mit vier Ampeln in Schöneberg eine Grüne Welle für RadfahrerInnen umgesetzt und plant bereits die zweite\footnote{\cite{BZ}}. Auch hier möchte man die Benutzung des Rades attraktiver machen und den Fahrradverkehr beschleunigen.
\subsection{Projekt Wolfsburger Welle}
Die \gls{VW}-Forschung initiierte in den 80er Jahren mit dem Projekt "'Wolfsburger Welle"' die ersten Untersuchungen zur "'Grünen Welle"' Informationen im Fahrzeug; mit der Idee, beim Annähern an eine Ampel die optimale Geschwindigkeit im Fahrzeug zu geben.\footnote{\cite{Welle}} "Dazu sendet die Ampelanlage ihren aktuellen Phasenzustand und eine Prognose für den nächsten Zustandwechsel an alle Fahrzeuge, die sich annähern. Der Fahrzeugcomputer setzt dann die aktuelle Fahrzeuggeschwindigkeit mit dem Abstand zur Ampel und der aktuellen Ampelphase in Bezug. Daraus wird errechnet, ob das Fahrzeug im Moment mit der grünen Welle ’mitschwimmt’ oder ob die Geschwindigkeit außerhalb des optimalen Bereichs liegt" \cite{MenschMaschine}.
\subsection{Projekt Travolution}
Im Sommer 2008 wurde das Projekt TRAVOLUTION (TRAffic \& eVOLUTION), von dem Amt für Verkehrsmanagement und Geoinformation der Stadt Ingolstadt, Audi AG\footnote{ Automobilhersteller, dem Volkswagen-Konzern zugehörig}, GEVAS Software\footnote{ Softwareunternehmen für Verkehrstechnik} und dem Lehrstuhl für Verkehrstechnik an der \gls{TUM} abgeschlossen. Es besteht aus den Teilprojekten \textsc{verkehrsadaptive Netzsteuerung mit Genetischen Algorithmen} und \textsc{Der informierte Fahrer}. Im Netzsteuerungsprojekt wurden 46 Lichtsignalanlagen in Ingolstadt mit der Netzsteuerungssoftware BALANCE ausgestattet, wodurch sie intelligent auf den Verkehr reagieren und die Schaltung an den Verkehr anpassen. 
\begin{figure}[H]  
    \centering  
    \includegraphics[width=0.7\textwidth]{audi-travolution}
    \label{fig:travolution}
    \caption[Projekt Travolution]{\gls{C2I}-Kommunikation: Der Bordcomputer zeigt die optimale Geschwindigkeit an, sodass die nächste Kreuzung ohne Halt überquert werden kann.\\ Quelle: {http://www.audiusanews.com/imagegallery/adhoc/12647//12647,12649,12652,12648,12651,12650,/travolution-promotes-eco-friendly-driving}}
\end{figure}
Ziel des zweiten Teilprojektes ist es, die Autofahrer über die Ampelphasen zu informieren. Die \gls{C2I}-Kommunikation mittels \gls{WLAN} und \gls{UMTS} umsetzend, senden mit Kommunikationsmodulen ausgestattete Ampeln die Grünphasen an den Bordcomputer der Autos, welcher widerum die Geschwindigkeit für ein reibungsloses Passieren errechnet\footnote{\cite{Travolution, AudiTravolution}}.\\ 
Fundierend auf Travolution wurden Folgeprojekte wie zum Beispiel das ebenfalls von Audi ins Leben gerufene "'Ampelinfo online"'. Über Mobilfunk ist in der \gls{C2X}-Anwendung das Auto mit dem zentralen Verkehrsrechner, welcher die Ampelanlagen steuert, vernetzt und visualisiert die entsprechenden Informationen im Bordcomputer. \footnote{\cite{Ampelinfo}}
%%% KOLIBRI %%%
\subsection{Projekt Kolibri}
In Bayern wurde im April 2011 das Pilot-Projekt "'KOLIBRI"' (Kooperative Lichtsignaloptimierung -- Bayrisches Pilotprojekt) mit den Teststrecken der B13 bei München mit sieben und der St2145 in der Nähe von Regensburg mit acht ampelgeregelten Kreuzungen gestartet. Gemeinsam untersuchten TRANSFER GmbH\footnote{ ein Beratungs- und Softwareunternehmen für Transport und Verkehr}, die \gls{BMW} Group, der Lehrstuhl für Ergonomie an der \gls{TUM} und die Oberste Baubehörde im Bayerischen Innenministerium die Funktionen und Auswirkungen eines Ampelassistenten außerhalb von Ortschaften\footnote{\cite{kolibri}, \cite{kolibriTUM}}. "'Per Mobilfunk übermittelt das Fahrzeug Rohdaten wie Zeit und genaue Position. Der Computer in der Zentrale kann daraus Informationen über die Verkehrslage, die Geschwindigkeit oder die Zahl der Ampelstopps und die Wartezeiten ermitteln, die dann als Korrekturgrößen wieder in die Steuerung der Lichtsignalanlage einfließen können."'(\cite{kolibriTUM}) Zusätzlich wurden die Fahrer sowohl fahrzeugintegriert\footnote{ On-Board-Computer} als auch via Smartphone über die Schaltung der nächsten Ampel informiert und erhielten Empfehlungen über die aktuelle Progressionsgeschwindigkeit. 
\subsection{Projekt Testfeld Telematik}
Ende des Jahres 2013 wurde in Wien das Projekt Testfeld Telematik -- Feldversuch zur Stärkung österreichischen Know-Hows im Bereich umweltverträglicher Mobilität erfolgreich abgeschlossen. Per \gls{C2X-glo}-Kommunikation bringt das Projekt unter Kooperative Dienste wie Ampelinformationen direkt ins Auto. Über Navigationssysteme, integrierte Systeme, Nachrüst-Plattformen oder mobile Endgeräte erreicht die FahrerInnen die Information der optimalen Geschwindigkeit sowie die Dauer der jeweiligen Ampelphase. Um an die Informationen zu kommen wurden unter anderem Kameras und Sensoren, beispielsweise als Induktionsschleife in die Fahrbahn eingelassen\footnote{\cite{Siemens}}.
%Die Kommunikation zwischen den beteiligten Objekten verlief über ITS-GS, WAVE, CALM-IR und Mobilfunk.
\begin{figure}[H]
    \centering
    \includegraphics[width=0.7\textwidth]{telematik}
    \label{fig:telematik}
    \caption[Projekt Testfeld-Telematik Ampelinformation]{"'Grüne Welle bei 50 km/h"'  Quelle: \cite{Telematik}}
\end{figure}
Andere Autohersteller wie \gls{BMW}, Volvo und Volkswagen kooperieren als Forschungsprojekt "'Car 2 Car Communication Consortium"' mit Testfeld Telematik, ebenfalls mit dem Ziel die Sicherheit an Kreuzungen zu verbessern. Im Auto installierte Sensoren kommunizieren mit Kameras und Scanner in der Ampel. Allerdings funktioniert das System nur mit dem ambitionierten Ziel, wenn alle Autohersteller zusammenarbeiten und sich auf den gleichen Standard einigen.\footnote{\cite{Siemens}}
%TOYOTA 
\\ \textbf{WO SOLL DER TOYOTA ABSCHNITT HIN??? :} \\
Auch Toyota hat ein System entwickelt, welches eine spezielle Infrastruktur an Kreuzungen, die Installation von Infrarot-Sendern, die mit dem Toyota-Navigationssystem kommunizieren erfordert. An roter Ampel werden die Fahrer über die verbleibende Wartezeit informiert. Die ausgestatteten Navigationssysteme wurden bis jetzt jedoch ausschließlich in Japan getestet.\footnote{\cite{Toyota}}\\
%
%%% Apps %%% 
%
\section{Apps}
Ampelassistenten als App sind relativ unproblematisch. Smartphones sind bereits mit einem \gls{GPS}-Empfänger ausgestattet und haben fast durchgängig Internetzugang. Die hier vorgestellten mobilen Anwendungen existieren bereits oder befinden sich in der Testphase.
%%% ENLIGHTEN %%%
\subsection{EnLighten}
EnLighten erkennt rote Ampeln und visualisiert die Dauer dieser Phase. Die mobile Anwendung nutzt \gls{GPS} zur Lokalisierung des Autos und verwendet ebenfalls die \gls{C2X}-Kommunikation zu Ampelphasenprognose.
\begin{figure}[H]
    \centering
    \includegraphics[width=0.7\textwidth]{EnLighten}
    \label{fig:Ampelsignalstatus}
    \caption[EnLighten]{Schnappschuss der Echtzeit Ampelsignalstatusprognose in Portland. Quelle: \cite{EnLighten}}
\end{figure}
Hierbei verbindet sich die App mit den Lichtsignalanlagen und beachtet dabei Komponenten wie die Höchstgeschwindigkeit, Fahrtrichtung und Tageszeit. Aufgrund von hohen Installationskosten und -Aufwand ist EnLighten erst in einigen amerikanischen Städten funktionstüchtig und verfügbar.
%%% SIGNAL GURU %%%
\subsection{Signal Guru}
Signal Guru wurde von den Wissenschaftlern des \gls{MIT} und der Universität von Princeton entwickelt. Die App errechnet über die Smartphones vieler Nutzer - welche miteinander kommunizieren -  die Wahrscheinlichkeit, wann eine Ampel grün wird und wie das eigene Fahrverhalten entsprechend anzupassen ist. Wie in Abbildung \ref{fig:AppSignalGuru} ist zu sehen ist, muss die eingebaute Kamera durch die Windschutzscheibe die Ampel registrieren. Bei Testläufen im Straßenverkehr vielen die Ergebnise bei statisch geschalteten Ampeln deutlich besser aus als bei angepassten Ampelschaltungen \footnote{\cite{SignalGuru}} 
\begin{figure}[H]
    \centering
    \includegraphics[width=0.7\textwidth]{SignalGuru}
    \caption[Signal Guru]{Signal Guru muss in der Lage sein die Ampel zu 'sehen'.  Quelle: \cite{SignalGuruPaper}}
    \label{fig:AppSignalGuru}
\end{figure}
\textit{Ob das auch in Deutschland funktioniert ist schwer zu sagen, da die Ampeln hierzulande so gesetzt sind, dass das Smartphone in der Pole-Position die Ampel evtl. nicht erfassen kann. Dies gilt es in der Entwicklungsarbeit zu testen und gegebenenfalls auszuarbeiten.}s
%%% AMPELMETER %%%
\subsection{Ampelmeter}
\textbf{= Lorem Ipsum: gibt/gabs es die App wirklich? \\}
Ampelmeter ist eine Anwendung, die eine Geschwindigkeitsempfehlung angibt, bei der man die in Fahrtrichtung nächste Ampel bei grün erreichen. Der zweite Anwendungsfall ist die Restrot- bzw. Restgrünanzeige. Da der timingbezogene Teil der Datenbank zum Startzeitpunkt noch leer ist, bedarf es der Mitarbeit der NutzerInnen.
%%% ERGIBNIS %%%
\section{Analyseergebnis}
Diese Beispiele zeigen deutlich, dass die Nachfrage nach Ampelassistenten -- mobil oder statisch -- steigt und auf dem Markt Anklang findet. Der Verkehr ist flüssiger, die Teilnehmer entspannter, die Luft sauberer. AutofahrerInnen sind schon lange nicht mehr allein auf der Straße und so gilt es, dieses erfolgreiche Konzept für alle VerkehrsteilnehmerInnen zu erweitern.

\chapter{Lösungsansätze}
Zur Umsetzung der beschriebenen Ampelinformationsanwendung kommen zwei Möglichkeiten in die engere Wahl. In diesem Kapitel werden die Realisierung durch eine \gls{Smartphone} \Gls{App} und die einer \gls{Arduino}-Anwendung gegenübergestellt. Zu beachten sind die Komponenten wie Sensorik, Internetverbindung, Stromversorgung und Darstellung der Informationen.\\
\begin{description}[leftmargin=0.7cm,style=nextline]
%GPS
  \item[\gls{GPS}] ~ Als Grundlage aller modernen Navigations- und Ortungssysteme im Bereich der Navigation ist \gls{GPS} für die Fahrradpositionsbestimmung obligatorisch. In einem \gls{Smartphone} ist ein \gls{GPS} Sensor inklusive, für eine \gls{Arduino}-Anwendung ein entsprechendes Modul vonnöten\footnote{ Vgl. \cite{arduino} S. 227}.\\
  % G-SENSOR
  \item[Beschleunigungssensor] ~ Der im \gls{Smartphone} integrierte Beschleunigungssensor ist ein Hardwaresensor der neben Neigung, Erschütterung, Vibration die Beschleunigung des Gerätes misst. Gerade bei Fahrten durch Tunnel ist dieser Sensor wichtig, da die Geschwindigkeitsberechnung mittels \gls{GPS}-Werten dort nicht möglich ist. Verdschiedene Beschleunigungssensoren gibt es auch für den \gls{Arduino}. Eine Platine deren Werte man nach Anklemmung auslesen kann\footnote{\url{http://bildr.org/2011/04/sensing-orientation-with-the-adxl335-arduino/}}.
\\
  %WWW
  \item[Internetverbindung] ~ Durch mobile Breitbandverbindung oder auch wahlweise per \gls{WLAN} ist beim \gls{Smartphone} eine Internet-Anbindung vorhanden. 
  Das \gls{Arduino}-Board benötigt vür die Internetverbindung eine Erweiterung um das Ethernet-Shield\footnote{ Vgl. \cite{arduino} S. 36}.\\
  %STROM
  \item[Stromversorgung] ~ Die Stromversorgung ist im \gls{Smartphone} durch den integrierten Akku gegeben. Die Laufzeit ist vom Typ abhängig, genügt jedoch für die alltägliche Radstrecke. Für die mobile Stromversorgung des \gls{Arduino}-Boards wird eine 9-Volt Batterie benötigt, die zusätzlichen Platz beansprucht und wassergeschützt und gut erreichbar angebracht werden muss. Sicher gibt es weiterhin die Möglichkeit den Strom aus dem Nabendynamo zu gewinnen -- dies bedarf jedoch zusätzlicher Arbeit. Außerdem sollte dann der \gls{Arduino} in der Lage sein Strom zu speichern, sodass die Anwendung beim Halt an der Ampel nicht ausschaltet.\\
  %DARSTELLUNG
  \item[Darstellung] ~ Ein Darstellungskonzept muss bei beiden Möglichkeiten erstellt werden. Auf dem \gls{Smartphone} ist besonders auf Erkennbarkeit bei schlechten Witterungsbedingungen, das ggf. spiegelnde Display berücksichtigend zu achten. Dafür sind aufgrund des vorhandenen Displays in gewisser Größe wesentlich mehr Informationen darstellbar. Als \gls{Arduino}-Anwendung sind lediglich ein paar helle \glspl{LED} am bzw. im Lenker erforderlich. Diese müssen doch eindeutig und intuitiv lesbar sein, um den vollen Informationsumfang zu gewähreisten.\\
 %Datenbankanbindung
 \item[Datenbankanbindung] ~ Ein weiterer wichtiger Aspekt ist die Datenbankanbindung. Mindestens die Position der Ampeln und die Phasen der Schaltpläne werden in einer Datenbankbank gespeichert und dort von der Anwendung angefragt und ausgewertet. Während man für die \gls{Arduino}-Anwendung eine Client-Server Architektur aufbauen und via Internet auf die Datenbank zugreifen muss, liefert Android die Datenbank SQLite mit.\\
\end{description}
\subsection*{Ergebnis}  
Auch wenn das Ampelinformationssystem als \gls{Arduino}-Anwendung minimalistischer ist sprechen die anderen Punkte dagegen. Beginnend bei der einfachen Stromversorgung bis hin zur vorhandenen Sensorik und internen Datenbank steht das \gls{Smartphone} weiter vorn und überzeugt durch seine Einfachheit in der Anwendung. Auch die vielen Erweiterungen die es bereits für das Fahrrad in Form einer mobilen Anwendung gibt, zeigen dass Lösungen für oben genannte Probleme wie die Nutzung bei schlechten Witterungsbedingungen existieren. Der Prototyp wird als Android Anwendung umgesetzt.

\chapter{\label{chap:grundlagen}Grundlagen}
Dieses Kapitel befasst sich mit sowohl den mathematischen als auch den technischen Grundlagen der zu behandelnden Thematik, welche für das weitere Verständnis der Arbeit beitragen.
% -------------------------------------------------
% TECHNISCHE GRUNDLAGEN
% -------------------------------------------------
\section{\label{sec:technGrundlagen}Technische Grundlagen}
Im folgenden Abschnitt werden Funktionsweise und Besonderheiten der verwendeten Technologien beschrieben. Es wird eine Smartphone Anwendung erstellt, deren Grundlage für die Implementierung die Software-Plattform Android ist.
% ANDROID
\subsection{Android}
Die Open-Source-Plattform Android umfasst das Betriebssystem, die Middleware sowie die wichtigsten Anwendungen(z.B. E-Mail-Client, SMS-Programm, Karten-Anwendung). Das Android \gls{SDK} stellt die Tools und \gls{API} zur Verfügung die erforderlich sind, Anwendungen auf der Android-Plattform implementieren zu können \cite{androidwww}\\
Mit dem Android \gls{SDK} können Android-Anwendungen mit der Programmiersprache Java entwickelt werden. Das \gls{SDK} enthält eine Reihe von Core-Bibliotheken, die die meisten Funktionen der Java-Core-Bibliotheken zur Verfügung stellen. Jede Android-Applikation läuft als ein eigener Prozess, eine eigene Instanz der sogenannten Dalvik Virtual Machine. Diese virtuelle Maschine (Dalvik VM) wurde so implementiert, dass auf einem Gerät mehrere Dalvik VMs effizient nebeneinander ausgeführt werden können. Das dazugehörige Dalvik Excecutable-Format wurde für geringeren Speicherverbrauch konzipiert. \footnote{https://developer.android.com/sdk/index.html}\cite{androidwww} \\ 
Mit dem Android \gls{NDK} existiert auch ein Werkzeug, mit dem Teile von ANwendungen in den Programmiersprachen C oder C++ implementiert werden können. Die Verwendung beider Sprachen bietet sich im Besonderen bei CPU-intensiven Operationnen wie zum Beispiel Signalverarbeitung oder Physik-Simulationen an. \footnote{https://developer.android.com/tools/sdk/ndk/index.html}
Die Folgende Abbildung zeige einen Überblick über die komplexe Androis-Systemarchitektur, welche nachfolgend nach \cite{android} kurz beschrieben werden.
%grafik
\paragraph{Linux Kernel: }
\paragraph{Bibliotheken: }
\paragraph{Android Runtime: }
\paragraph{Application Framework: }
\paragraph{Application Layer: }
\begin{itemize}
	\item Activity Manager und Fragment Manager
	\item Views
	\item Notification Manager
	\item Resource Manager
	\item Intents
\end{itemize}

% SQLite
\subsection{Die Datenbank SQLite}
% MOBILE SENSING
\subsection{Mobile Sensorik} 
\subsubsection{Mobile Sensorik unter Android}
\subsubsection{Geolokation mittels \gls{GPS}}
\subsubsection{G-Sensor}
Blabla Beschleunigungssensor...
\clearpage
% -------------------------------------------------
% MATHEMATISCHE GRUNDLAGEN
% -------------------------------------------------
\section{\label{sec:mathGrundlagen}Berechnung der Geschwindigkeitsempfehlung}
Präsentiert das System während der Anwendung eine Geschwindigkeitsempfehlung, ist diese abhängig von der Fahrtgeschwindigkeit und vom Abstand zur Ampel. Angenommen die Progressionsgeschwindigkeit $v$ wird zum Zeitpunkt $t_{1}$ ermittelt, die \gls {LSA} schaltet zum Zeitpunt $t_{2}$ auf Rot und Abstand zur Ampel beträgt $s$, dann gilt: \\
\[ v = \frac{s}{t_{2} - t_{1}} \] \\
Die von der Berliner Verkehrsleitzentrale zur Verfügung gestellten Ampelschaltpläne und Position der angesteuerten Ampel dienen als Grundlage dieser Berechnung und sind aus der Datenbank zu holen. Die aktuelle Position des Fahrrads wird vom \gls{GPS} Sensor des \glspl{Smartphone} ermittelt und daraus der Abstand zur Ampel errechnet. Die Abbildung \ref{fig:vst} soll die Berechnungsgrundlagen veranschaulichen: 
\begin{figure}[H]  
    \centering  
    \includegraphics[width=1\textwidth]{vst}     
    \caption[Berechnung Progressionsgeschwindigkeit]{Veranschaulichung der Berechnung}
    \label{fig:vst}
\end{figure}
Um die ensprechende \gls{LSA} während der Grünphase zu passieren, muss letztendlich die empfohlene Geschwindigkeit $v$ eingehalten werden.

\chapter{\label{chap:szenarien}Szenarien im Ampelbereich}
Alle in Kapitel \ref{chap:state} angeführten Studien zu Ampelinformationssystemen und Konzepte zu Fahrrad- erweiterungen haben die Gemeinsamkeit des selbstkontrollierten Fahrverhaltens der FahrerInnen. Die Verkehrslage und die Ampelsituation sind trotzdem zu beachten. Ausgesprochen werden lediglich Empfehlungen, die möglichst intuitiv und schnell vermittelt werden können.\\ 
Prinzipiell sollte die Anwendung in der Lage sein, bei einer Annäherung an eine Ampel die passende Empfehlung oder Handlungsaufforderung anzuzeigen, die sich aus den Szenarien ergeben. Folgende Situationen können eintreten:\\
\begin{figure}[H]  
    \centering  
    \includegraphics[width=1\textwidth]{Szenarien} 
    \grayRule
    \caption[Szenarien]{Szenarien im Ampelbereich}
    \label{fig:szenarien}
\end{figure} \vspace{17pt}
\begin{description}[leftmargin=0.7cm,style=nextline]
\clearpage
% ROT
\item[Szenario R1:] 
Die Fahrradfahrerin oder der Fahrradfahrer nähert sich einer aktuell roten Ampel. Die Anwendung zeigt den Countdown der Restrotzeit an und empfiehlt, langsamer zu fahren, um die Ampel ohne anzuhalten passieren zu können.  \\
\item[Szenario R2:] 
Die Fahrradfahrerin oder der Fahrradfahrer nähert sich einer aktuell roten Ampel. Die Anwendung zeigt an, dass eine Weiterfahrt bei gleichbleibender Geschwindigkeit gewährleistet ist. Es besteht kein Aktionsbedarf. \\
\item[Szenario R3:] 
Die Fahrradfahrerin oder der Fahrradfahrer nähert sich einer aktuell roten Ampel. Die Anwendung meldet, das Erreichen der Grünphase ist auch bei Geschwindigkeitsreduktion nicht möglich.\\%in jedem Fall
% GRÜN 
\item[Szenario G1:] 
Die Fahrradfahrerin oder der Fahrradfahrer nähert sich einer aktuell grünen Ampel. Die Anwendung zeigt den Countdown der Restgrünzeit an und empfiehlt schneller zu fahren, um die Ampel ohne anzuhalten passieren zu können.\\
\item[Szenario G2:] 
Die Fahrradfahrerin oder der Fahrradfahrer nähert sich einer aktuell grünen Ampel. Die Anwendung zeigt an, dass eine Weiterfahrt bei gleichbleibender Geschwindigkeit gewährleistet ist. Es besteht kein Aktionsbedarf.\\ 
\item[Szenario G3:] 
Die Fahrradfahrerin oder der Fahrradfahrer nähert sich einer aktuell grünen Ampel. Die Anwendung meldet, das Anhalten bei einer auf Rot umspringenden Ampel ist in jedem Fall erforderlich, da eine zu hohe Geschwindigkeit zum Erreichen der noch grünen Ampel erforderlich wäre.\\
% Verkehrsabgängig oder aus
\item[Szenario V1:] 
Die Fahrradfahrerin oder der Fahrradfahrer nähert sich einer ausgeschalteten Ampel. Es gibt weder Grün- noch Rotphasen. Das System zeigt an, dass die aktuelle Verkehrslage beurteilt und entsprechend gehandelt werden sollte.\\ 
\end{description}
\clearpage
\centerline{\grayRule}
Die obigen Szenarien gehen von einer gleichbleibenden Ampelschaltung aus. Insbesondere in Großstädten gibt es jedoch viele Faktoren, die die Ampelschaltung beeinflussen können, indem sie Grün anfordern. Deshalb kann hier Szanario V2 eintreten. Da keine Echtzeitdaten mit speziellen Technologien zur Verfügung stehen, sollte die Anwendung für die Berechnungen ausschließlich die intervallgesteuerten \glspl{LSA} berücksichtigen.\\
\centerline{\grayRule}
\begin{description}[leftmargin=0.7cm,style=nextline]
\item[Szenario V2:] 
Die Fahrradfahrerin oder der Fahrradfahrer nähert sich einer verkehrsabhängigen Ampel. Das bedeutet, die Schaltung ist unregelmäßig und die Wahrscheinlichkeit des Zutreffens der Vorhersage zu gering. Die Anwendung zeigt also an, dass es ihr nicht möglich ist, eine Vorhersage zu treffen.\\ 
\end{description} %\vspace{27pt}
% ERGEBNIS
\subsection*{Ergebnis}
Da die Szenarien \textit{R2} und \textit{G2}, die Szenarien \textit{R3} und \textit{G3} sowie die Szenazien\textit{V1} und \textit{V2} zusammengefasst werden können, ergeben sich aus den acht Szenarien im Ampelbereich die fünf nun aufgezählten Fälle.
\begin{itemize}
	\item Fall a: Anhalten bei Rot in jedem Fall erforderlich
	\item Fall b: Weiterfahrt mit gleichbleibender Geschwindigkeit möglich.\\ Kein Aktionsbedarf
	\item Fall c: Weiterfahrt bei Verlangsamung möglich
	\item Fall d: Weiterfahrt bei Beschleunigung möglich
	\item Fall e: Keine Vorhersage möglich
\end{itemize}
Im weiteren Verlauf dieser Arbeit wird unter anderem beschrieben, wie diese Fälle in die Konzeption der zu entwickelnden Anwendung eingebunden werden.

\chapter{Die Anforderungsdefinition}
Dieses Kapitel soll durch Untersuchung helfen, Vorstellung für die Anforderungen an Ampelhinweissystem -App zu bekommen. Begriff Persona wird eingeführt, erklärt, entwickelt.
Mit Hilfe dieser Personas ... werden analysen gemacht + kritisch beurteilt.
Den Abschluss bildet das Ergebnis dieser Untersuchung in Zusammenfassung der wichtigsten Anforderungen an eine App.
\section{Personas}
\subsection*{Einleitung}
Im heutigen High-Technologie Zeitalter ist gerade die Benutzbarkeit bei der Entwicklung eines Produktes ein wichtiger Faktor, der von Software-Entwicklern beachtet werden muss. Die Anforderungen der Nutzer stehen dabei im Mittelpunkt. Es geht in erster Linie darum, jene zufrieden zu stellen und nicht nur Interesse, sondern auch Begeisterung beim potentiellen Kunden zu wecken. Verschiedene Methoden, diese Anforderungen besser zu identifizieren und erfüllen zu können, haben sich bereits verbreitet und basieren meistens auf einer präzisen Darstellung der Nutzer. Eine erprobte Methode hat der Software-Entwickler Alan Cooper eingeführt: Personas oder Archetypen von Nutzern.
\subsection*{Definition}
Fokus auf Gruppe spezifischer Nutzer bekommen blabla
\subsection*{Grund für Personas}
Effizienz mit Personas
\subsection*{Prototyp: Personas}
Um eine mögliche Anforderungsanalyse erarbeiten zu können, ist die
Wahl auf Personas, als Kriterium der Anforderung von Zielnutzern, gefallen. Auf den nachfol-
genden Seiten sind vier verschiedene Personas in einem übersichtlichen Tabellenprofil aufge-
listet.
\section{Funktionalität}
\section{Diagramme}
\subsection*{Sequenzdiagramm}
\subsection*{Ablaufdiagramm}
\subsection*{UML -- Diagramm}
\section{Die graphische Oberfläche}
   
\chapter{\label{chap:entwurf}Konzeption}
% ### Achitektur ###
\section{Architektur}
Klassenstruktur..
\subsection{Navigation}
% ### Design ###
\section{Das Design}
Hohe Kontraste wegen Wetter.\\
einfache UI .\\
Minimalistisch sodass ein Blick genügt und man nicht abgelenkt wird.\\
Angezeigt wird: Geschwindigkeit schneller oder langsamer, 
Restrotanzeige, grüne Welle.
\begin{figure}[H]
        \centering
           \begin{subfigure}[t]{0.23\textwidth}
                \includegraphics[width=\textwidth]{stop1}
                \caption[Systemzustand d]{keine Weiterfahrt möglich}
                \label{fig:stop}
        \end{subfigure}
           \hfill 
              \begin{subfigure}[t]{0.23\textwidth}
                \includegraphics[width=\textwidth]{yeah1}
                \caption[Systemzustand c]{Kein Aktionsbedarf}
                \label{fig:yeah}
        \end{subfigure}
           \hfill
        \begin{subfigure}[t]{0.23\textwidth}
                \includegraphics[width=\textwidth]{langsamer}
                \caption[Systemzustand a]{Weiterfahrt durch Verlangsamung möglich}
                \label{fig:langsamer}
        \end{subfigure}
        \hfill
        \begin{subfigure}[t]{0.23\textwidth}
                \includegraphics[width=\textwidth]{schneller}
                \caption[Systemzustand b]{Weiterfahrt durch Beschleunigung möglich}
                \label{fig:schneller}
        \end{subfigure}     
        \caption[Systemzustände im Ampelbereich]{Umsetzung des Designs anhand der Systemzustände}
        \label{fig:mockup}
\end{figure} 
\subsection{Anzeigeelemente}
\subsubsection{Geschwindigkeit}
\subsubsection{Ampeln}
\subsubsection{Informationen}
\section{Ampeldatenanfrage und Auswertung}
\subsection{Sensoren}
\subsubsection{GPS} wird gebraucht für...
\section{Theorie}
Um die korrekte Umsetzung des Prototyps zu ermöglichen, müssen zunächst einmal prinzipielle
Theorien und Hintergründe diesen betreffend betrachtet werden.
grundlegendes Wissen über geographische Koordinaten sowie mathematische Voraussetzungen im Umgang mit diesen, müssen zur Ideenverwirklichung berücksichtigt werden.
\subsection{Die Berechnung der Entferung}
\subsection{Die Berechnung der Ankunft in Abhängigkeit der Geschwindigkeit}
\subsection{Die Anzeige der Restrotanzeige}
         
\chapter{\label{chap:implementierung}Der Prototyp}
In diesem Kaptitel wird die nach dem in Kapitel \ref{chap:entwurf} präsentierten Lösungswegs die detaillierte Beschreibung der technischen Realisierung der Anwendung vorgestellt.\\
Nach der Erklärung der Konfigurationsdateien wird auf die Umsetzung der Szenarien eingegangen. Im Zuge dessen werden die implementierten Algorithmen vorgestellt, wobei sich der erste mit dem Auffinden der nächsten relevanten Ampel befasst und der zweite die empfohlene Geschwindigkeit berechnet.
\section{Die Manifest- und build.gradle-Datei}
Das Android-Manifest dient der Festlegung wichtiger Eigenschaften der Anwendung und gehört zu jedem Android-Projekt. (+ gradle...)\\

Die \gls{XML}-Datei (\texttt{AndroidManifest.xml}) ist im Hauptverzeichnis des Projekts zu finden und ist im Listing \ref{lst:manifest} abgebildet. \\
In der zweiten Zeile wird hier der Paketname des Programms festgelegt. 
Im \texttt{application}-Tag werden Variablen gesetzt, die das in dargestellte Icon und den Namen der Anwendung definieren. Darüber hinaus wird hier die \gls{Activity} der Applikation definiert. Zuerst wird der Name der \gls{Activity} gesetzt. Die Variable \texttt{screenOrientation} legt das Format der Anzeige fest und verhindert ein automatisches Drehen des Bildschirms. Im \texttt{intent-filter}-Tag dass diese Activity beim Start der App ausgeführt wird. Hätte die Anwendung über mehrere \glspl{Activity} implementiert, würden die anderen ebenfalls hier aufgeführt werden.\\
Unterhalb des \texttt{application}-Tags, in Zeile 17, werden nun die Berechtigung des \gls{GPS}-Zugriffs der Applikation, um Standortdaten, also die jeweiligen geographischen Kordinaten des Endgeräts zu beziehen gesetzt.
\begin{center}
\rule{35em}{0.5pt} \lstinputlisting[language=XML, firstline=2, lastline=21, caption={AndroidManifest.xml}, label=lst:manifest]{code/manifest.xml}
 \rule{35em}{0.5pt}
\end{center}
Für welche Android Versionen die Anwendung geschrieben wurde (\texttt{targetSdkVersion}) und das minimale \gls{API}-Level der Anwendung, also unter welcher Version die App noch ausgeführt werden kann:\\
\begin{center}
\rule{35em}{0.5pt} \lstinputlisting[language=JSON, firstline=6, lastline=14,  caption={build.gradle}, label=lst:gradle]{code/build.gradle} \rule{35em}{0.5pt}
\end{center}
Hier wird der Name der \gls{Activity}-Klasse gesetzt und im \texttt{intent-filter}-Tag festgelegt, dass diese \gls{Activity} als \texttt{MainActivity} beim Start der Anwendung ausgeführt wird. \textit{label:android:?}
\section{MainActivity-Klasse}
\section{Umsetzung Szenarien}
\subsection{?Einlesen der Ampeldaten?}
\subsection{Ermittlung der nächsten Ampel}
\subsection{Algorithmus für die Geschwindigkeitsempfehlung}

%
% TEST
%
\chapter{Evaluierung}
\section{Systemtest}
ist das GPS schnell/genau genug fürs Radfahren?
Optimierung ggf. umsetzen\\
Personenbezogene Daten aufnehmen? Höchstgeschwindigkeit, maximale Beschleunigung

\chapter{Evaluation}
Um die Funktionalität des Prototyps zu erproben, wurden folgende Testgeräte gewählt. Es handelt sich hierbei um Geräte mit unterschiedlichen Bildschirmgrößen und Android-Versionen.\\ 
\begin{table}[H]
\centering	
	\begin{tabular}{@{}>{\columncolor[HTML]{ECF4FF}}l ll@{} p{0.4\textwidth}p{0.2\textwidth}p{0.2\textwidth}} \toprule	
\multicolumn{1}{c}{\cellcolor[HTML]{ECF4FF}\textbf{Testgerät}} 
& \multicolumn{1}{c}{\cellcolor[HTML]{ECF4FF}\textbf{Android-Version}} 
& \multicolumn{1}{c}{\cellcolor[HTML]{ECF4FF}\textbf{Bildschirm}} \\ \hline
% GALAXY NOTE 2
\multicolumn{1}{l}{\cellcolor[HTML]{ECF4FF}\textbf{Samsung Galaxy Note 2}} 
& \multicolumn{1}{p{0.2\textwidth}}{4.4.2}
& \multicolumn{1}{p{0.2\textwidth}}{1280 x 720 Pixel} \\ \midrule
% GALAXY NEXUS
\multicolumn{1}{l}{\cellcolor[HTML]{ECF4FF}\textbf{Samsung Galaxy Nexus}} 
& \multicolumn{1}{p{0.2\textwidth}}{4.3}
& \multicolumn{1}{p{0.2\textwidth}}{1280 x 720 Pixel} \\ \midrule
% Samsung Nexus S
\multicolumn{1}{l}{\cellcolor[HTML]{ECF4FF}\textbf{Samsung Nexus S}} 
& \multicolumn{1}{p{0.2\textwidth}}{4.1.2}
& \multicolumn{1}{p{0.2\textwidth}}{800 x 480 Pixel}\\ \midrule
% LG Nexus 4
\multicolumn{1}{l}{\cellcolor[HTML]{ECF4FF}\textbf{LG Nexus 4}} 
& \multicolumn{1}{p{0.2\textwidth}}{4.4.4}
& \multicolumn{1}{p{0.2\textwidth}}{1280 x 768 Pixel}\\ \midrule
% LG Nexus 5
\multicolumn{1}{l}{\cellcolor[HTML]{ECF4FF}\textbf{LG Nexus 5}} 
& \multicolumn{1}{p{0.2\textwidth}}{5.0}
& \multicolumn{1}{p{0.2\textwidth}}{1920 x 1080 Pixel}\\ \midrule
%  Motorola RAZR Maxx
\multicolumn{1}{l}{\cellcolor[HTML]{ECF4FF}\textbf{Motorola RAZR Maxx}} 
& \multicolumn{1}{p{0.2\textwidth}}{4.0.4}
& \multicolumn{1}{p{0.2\textwidth}}{540 x 960 Pixel}\\ \midrule
% HTC Desire HD
\multicolumn{1}{l}{\cellcolor[HTML]{ECF4FF}\textbf{HTC Desire HD}} 
& \multicolumn{1}{p{0.2\textwidth}}{2.3.5}
& \multicolumn{1}{p{0.2\textwidth}}{480 x 800 Pixel}\\ \bottomrule
\end{tabular}
\caption{Testgeräte}
\rule{35em}{0.5pt}
\label{tab:geräte}
\end{table}
fgffdjhi
hziuhziu

\section{Systemtest}

ist das GPS schnell/genau genug fürs Radfahren?
Optimierung ggf. umsetzen\\
Personenbezogene Daten aufnehmen? Höchstgeschwindigkeit, maximale Beschleunigung
\section{Testergebnisse}

\chapter{\label{chap:fazit}Ergebnis und Ausblick}
\section{Ampelhinweissystem}
\section{Ausblick}

%
% Verzeichnisse
%
%% Ein kleiner Abstand zu den Kapiteln im Inhaltsverzeichnis (toc)
\addtocontents{toc}{\protect\vspace*{\baselineskip}}
\include{extra/Acronym}
\printglossary[type=\acronymtype, title=Abkürzungen, style=super]
\printglossary[type=main,style=altlist]
% 
%Abbildungsverzeichnis
%
\clearpage
\addcontentsline{toc}{chapter}{Abbildungsverzeichnis}
\listoffigures
 \renewcommand\lstlistlistingname{Quellcodeverzeichnis} %Inhaltsverzeichnistitel = Inhalt
\lstlistoflistings%quellcodeverzeichnis
% Literaturverzeichnis
\addcontentsline{toc}{chapter}{Literaturverzeichnis}
\bibliographystyle{alphadin} 
\bibliography{extra/literatur}
\nocite{stvo}
% Anhang
\clearpage
\appendix
%\section{Eidesstattliche Erkl�hrung}
Ich versichere, dass ich die vorliegende Bachelorarbeit mit dem Titel 
Ampelphasen-Informationssystem f�r FahrradfahrerInnen auf Grundlage persistenter geo- und zeitbasierter Daten ohne Hilfe Dritter und ohne Benutzung anderer als der angegebenen Hilfsmittel angefertigt habe; die aus fremden Quellen direkt oder indirekt �bernommenen Gedanken sind als solche kenntlich gemacht. Die Arbeit wurde bisher in gleicher oder hnlicher Form in keiner anderen Pr�fungsbeh�rde vorgelegt und auch noch nicht ver�ffentlicht.\\[4cm]

\hspace{2cm} Berlin, den \today{} \hfill Unterschrift \hspace{2cm}

\addcontentsline{toc}{chapter}{Anhang}
%\include{extra/Eidesstattliche}
\end{document}
